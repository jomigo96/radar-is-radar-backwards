% arara: xelatex: { synctex: yes, shell: yes }
% arara: biber
% arara: xelatex: { synctex: yes, shell: yes }

\RequirePackage[l2tabu, orthodox]{nag}
\documentclass[purist,portuguese]{ist-report}

% -- Texto e codificação
\usepackage{anyfontsize}
\usepackage{pdflscape}

% -- Funções matemáticas extra
\usepackage{siunitx}

% -- Símbolos extra
\usepackage{amssymb}
\usepackage{textcomp}
\usepackage{gensymb}
\usepackage{cancel}

% -- Bibliografia
\usepackage[
	backend = biber,
	style = alphabetic,
	sorting = ynt
	]{biblatex}
\usepackage{fvextra}
\usepackage{csquotes}
\usepackage{booktabs}

% --  Definições de imagens
\graphicspath{{graphics/}}
\usepackage{caption}
\usepackage{subcaption}
\usepackage{afterpage}
\usepackage{tabularx}

% -- Desenhar circuitos elétricos e lógicos
\usepackage{tikz}
\usepackage{pgfplots}
\usepackage{circuitikz}
\usetikzlibrary{arrows.meta,positioning,patterns,graphs}
\pgfplotsset{compat=1.16}
\pgfplotsset{table/search path = {data}}
\pgfplotsset{/pgf/number format/use comma,}

% -- Integrar código fonte
\usepackage{minted}
\usepackage{verbatim}

% -- Misc
\usepackage{todonotes}

\addbibresource{main.bib}

\author{Daniel de Schiffart \\ \texttt{81479} \and João Gonçalves \\ \texttt{81040}}
\title{Radar MIMO}
\subtitle{}
\course{Mestrado Integrado em Engenharia Aeroespacial}
\subject{Sistemas de Radar}

\begin{document}

\makecover

\todo{Não tenho spellcheck, fazer isso no fim!}
\todo{É preciso arranjar muitas imagens, mas não encontro nada de jeito.}

{\hypersetup{linkcolor={black}} \tableofcontents}

\newpage

\begin{abstract}
	Neste trabalho vamos tratar de radares multi-estáticos MIMO, acrónimo para \textit{Multiple-Input Multiple-Output}.
\end{abstract}

\section{Introdução}

Este trabalho é baseado em \cite{davis2015mimo} e \cite{li2018mimo}.\todo{Obviamente que não dizemos isto assim, mas o texto vem 100\% dali, não podemos citar a cada parágrafo.}

Neste trabalho, é apresentada uma visão global sobre o radar MIMO, detalhando as suas principais características, história, desenvolvimentos em curso e aplicações.

\section{Definição}

Um sistema de radar ativo emite energia eletromagnética para analisar o ambiente.
Um radar MIMO usa vários elementos emissores e transmite várias formas de onda não correlacionadas, de maneira que um recetor pode atribuir a cada sinal o seu emissor. 
Em termos práticos, os emissores podem estar substancialmente afastados, designando-se radar MIMO não coerente ou estatístico, por razões estudadas mais à frente.
Se, por outro lado, os emissores se encontrarem próximos uns dos outros, possivelmente na mesma plataforma ou agregado, então o sistema é designado de radar MIMO coerente.

\subsection{MIMO coerente e não coerente}

No radar MIMO \textbf{coerente}, a colocação das antenas não difere do radar convencional, tal que as antenas de emissão e receção partilham o ângulo de observação a um alvo distante.
Uma consequência é que a RCS (\textit{radar cross-section}) será praticamente constante

Este tipo pode ser visto como uma generalização do radar monoestático.
\todo{As fontes estão bué confusas aqui, mas isto é importante}

No radar MIMO \textbf{não coerente}, os transmissores e receptores são necessariamente antenas separadas, colocadas em diferentes posições numa certa área, tal que os ângulos de observação ao alvo são diferentes para cada elemento.

\section{Perspetiva Histórica}

Em telecomunicações, técnicas MIMO são usadas desde os anos 90, pois apresentam um conjunto de vantagens em relação ao tradicional SISO (\textit{Single-input Single-output}).
Em primeiro lugar, aumenta a eficiência espectral, isto é, a quantidade de informação transmitida por segundo e por hertz para uma potência fixa. 
Além disso, melhora-se a robustez ao desvanescimento, pela informação poder ser enviada de emissores não correlacionados.

Em contrapartida à metodologia SISO em que se transmite toda a energia num só percurso de comunicação, as comunicações MIMO permitem explorar a diversidade espacial, já que o desvanescimento actua independentemente para cada transmissor.
Assim, o SNR (\textit{signal-to-noise ratio}) no recetor não depende dramaticamente do desvanescimento dos canais individuais.

Em essência, para comunicações em MIMO, a informação transmitida é desconhecida, mas os canais são conhecidos, e até estáticos. Para o radar MIMO, as formas de onda emitidas são conhecidas, mas o ambiente, como a presença de \textit{clutter} ou \textit{jamming}, não o são.

A motivação para o aplicar o conceito MIMO a radares foi mesmo a ideia de diversidade espacial, que permitiria ultrapassar a sintilação dos alvos através de transmissões não correlacionadas.
Contudo, a ideia não se restringe ao uso de vários elementos radiadores.
De facto, agregados de fase \todo{É esta a tradução de phased arrays?} são usados à muito tempo para detectar alvos pequenos a grandes distâncias.
A diferença é que neste caso os sinais de saída só diferem na fase entre si de maneira a guiar um lobo de ganho elevado. 
Estes sinais são então correlacionados e não oferecem mais nenhum grau de liberdade.

\todo{Mais a ser escrito aqui}

O primeiro protocolo de radar MIMO foi desenvolvido nos anos 90, tem o nome de \textit{Synthetic pulse and Antenna Radar}, usa um agregado que alimenta cada elemento de antena com um sinal específico, tal que o recetor é irradiado com vários lobos. 



\section{Pesquisa em radar MIMO}

\printbibliography
\listoftodos
\end{document}
