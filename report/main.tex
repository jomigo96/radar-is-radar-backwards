% arara: xelatex: { synctex: yes, shell: yes }
% arara: biber
% arara: xelatex: { synctex: yes, shell: yes }

\RequirePackage[l2tabu, orthodox]{nag}
\documentclass[purist,portuguese]{ist-report}

% -- Texto e codificação
\usepackage{anyfontsize}
\usepackage{pdflscape}

% -- Funções matemáticas extra
\usepackage{siunitx}

% -- Símbolos extra
\usepackage{amssymb}
\usepackage{textcomp}
\usepackage{gensymb}
\usepackage{cancel}

% -- Bibliografia
\usepackage[
	backend = biber,
	style = alphabetic,
	sorting = ynt
	]{biblatex}
\usepackage{fvextra}
\usepackage{csquotes}
\usepackage{booktabs}

% --  Definições de imagens
\graphicspath{{graphics/}}
\usepackage{caption}
\usepackage{subcaption}
\usepackage{afterpage}
\usepackage{tabularx}

% -- Desenhar circuitos elétricos e lógicos
\usepackage{tikz}
\usepackage{pgfplots}
\usepackage{circuitikz}
\usetikzlibrary{arrows.meta,positioning,patterns,graphs}
\pgfplotsset{compat=1.16}
\pgfplotsset{table/search path = {data}}
\pgfplotsset{/pgf/number format/use comma,}

% -- Integrar código fonte
\usepackage{minted}
\usepackage{verbatim}

% -- Misc
\usepackage{todonotes}

\addbibresource{main.bib}

\usepackage{newtxsf}
%\usepackage[math]{iwona}

\author{Daniel de Schiffart \\ \texttt{81479} \and João Gonçalves \\ \texttt{81040}}
\title{Radar MIMO}
\subtitle{}
\course{Mestrado Integrado em Engenharia Aeroespacial}
\subject{Sistemas de Radar}
\date{Dezembro de 2018}

\begin{document}

\makecover

\todo{Não tenho spellcheck, fazer isso no fim!}
\todo{É preciso arranjar muitas imagens, mas não encontro nada de jeito.}

{\hypersetup{linkcolor={black}} \tableofcontents}

\newpage

\begin{abstract}
	Neste trabalho vamos tratar de radares multi-estáticos MIMO, acrónimo para \textit{Multiple-Input Multiple-Output}.
\end{abstract}

\section{Introdução}

Este trabalho é baseado em \cite{davis2015mimo} e \cite{li2018mimo}.\todo{Obviamente que não dizemos isto assim, mas o texto vem 100\% dali, não podemos citar a cada parágrafo.}

Neste trabalho, é apresentada uma visão global sobre o radar MIMO, detalhando as suas principais características, história, desenvolvimentos em curso e aplicações.

Uma nota sobre nomenclatura, ao longo deste trabalho usam-se símbolos minísculos a negrito para designar vetores, como $\mathbf{x}(t)$, maiúsculos a negrito para designar matrizes, como $\mathbf{H}(t)$, e os símbolos a fonte normal são, a não ser que referido o contrário, unidimensionais.

\section{Definição}

Um sistema de radar ativo emite energia eletromagnética para analisar o ambiente.
Um radar MIMO usa vários elementos emissores e transmite várias formas de onda não correlacionadas, de maneira que um recetor pode atribuir a cada sinal o seu emissor. 
Pode portanto ser classificado como uma subdivisão de radar multiestático, em que as formas de onda utilizadas não são coerentes, tal que em cada pode ser feita a distinção da fonte dos sinais recebidos.

Em termos práticos, os emissores podem estar substancialmente afastados, designando-se radar MIMO não coerente ou estatístico, por razões estudadas mais à frente.
Se, por outro lado, os emissores se encontrarem próximos uns dos outros, possivelmente na mesma plataforma ou agregado, então o sistema é designado de radar MIMO coerente.

\subsection{MIMO coerente e não coerente}

No radar MIMO \textbf{coerente}, a colocação das antenas não difere do radar convencional, tal que as antenas de emissão e receção partilham o ângulo de observação a um alvo distante.
Uma consequência é que a RCS (\textit{radar cross-section}) será praticamente constante.

Este tipo pode ser visto como uma generalização do radar mono-estático.
\todo{As fontes estão bué confusas aqui, mas isto é importante}

No radar MIMO \textbf{não coerente}, os transmissores e receptores são necessariamente antenas separadas, colocadas em diferentes posições numa certa área, tal que os ângulos de observação ao alvo são diferentes para cada elemento.

\section{Perspetiva Histórica}

Em telecomunicações, técnicas MIMO são usadas desde os anos 90, pois apresentam um conjunto de vantagens em relação ao tradicional SISO (\textit{Single-input Single-output}).
Em primeiro lugar, aumenta a eficiência espectral, isto é, a quantidade de informação transmitida por segundo e por hertz para uma potência fixa. 
Além disso, melhora-se a robustez ao desvanecimento, pela informação poder ser enviada de emissores não correlacionados.

Em contrapartida à metodologia SISO em que se transmite toda a energia num só percurso de comunicação, as comunicações MIMO permitem explorar a diversidade espacial, já que o desvanecimento actua independentemente para cada transmissor.
Assim, o SNR (\textit{signal-to-noise ratio}) no recetor não depende dramaticamente do desvanecimento dos canais individuais.

Em essência, para comunicações em MIMO, a informação transmitida é desconhecida, mas os canais são conhecidos, e até estáticos. Para o radar MIMO, as formas de onda emitidas são conhecidas, mas o ambiente, como a presença de \textit{clutter} ou \textit{jamming}, não o são.

A motivação para o aplicar o conceito MIMO a radares foi mesmo a ideia de diversidade espacial, que permitiria ultrapassar a cintilação dos alvos através de transmissões não correlacionadas.
Contudo, a ideia não se restringe ao uso de vários elementos radiadores.
De facto, agregados de fase \todo{É esta a tradução de phased arrays?} são usados à muito tempo para detectar alvos pequenos a grandes distâncias.
A diferença é que neste caso os sinais de saída só diferem na fase entre si de maneira a guiar um lobo de ganho elevado. 
Estes sinais são então correlacionados e não oferecem mais nenhum grau de liberdade.

\todo{Mais a ser escrito aqui}

O primeiro protocolo de radar MIMO foi desenvolvido nos anos 90, tem o nome de \textit{Synthetic pulse and Antenna Radar}, usa um agregado que alimenta cada elemento de antena com um sinal específico, tal que o recetor é irradiado com vários lobos. 



\section{Pesquisa em radar MIMO}

\section{Radar MIMO}

\subsection{Princípos de comunicações MIMO e radar}

A meta de qualquer sistema de radar é inferir uma propriedade do ambiente a partir das características das ondas eletromagnéticas transmitidas.
Os sistemas de radar são normalmente modelados linearmente, com a entrada sendo o sinal irradiado, e a saída o sinal recebido. 
Para o caso SISO, denomeia-se $x(t)$ a representação complexa do sinal de entrada transportado na frequência $\omega_c$, $(t)$ a saída, $h(t)$ a resposta impulsiva e $v(t)$ o ruído no recetor.
A resposta pode ser dada por:
\begin{align}
  y(t) = \int_0^\infty h(\tau)e^{-i\omega_c\tau}(t-\tau)d\tau + v(t).
  \label{eq:linearsiso}
\end{align}

No caso das telecomunicações, $h(t)$ é função do canal de comunicações, enquanto que para o problema de radar monoestático pode ser chamada de \textit{range profile}.
Para qualquer dos casos, $h(t)$ pode ser construído pelo sinal medido de um certo número de recetores, que chegam com o mesmo atraso $\tau$, mas com diferentes ângulos de chegada $\theta$:
\begin{align}
  h(\tau) = \int_{-\pi}^{\pi}\alpha(\tau,\theta)d\theta.
  \label{eq:profile}
\end{align}
%Para o radar, o objetivo pode ser estimar este par ângulo/atraso.

No caso MIMO, é usado um conjunto de sinais de entrada, $\mathbf{x}(t)\in\mathbb{C}^M$, e um conjunto de sinais de saída, $\mathbf{y}(t)\in\mathbb{C}^N$.
O modelo da equação \ref{eq:linearsiso} pode ser expandido para:
\begin{align}
  \mathbf{y}(t) = \int_0^\infty \mathbf{H}(\tau)e^{-i\omega_C \tau}\mathbf{x}(t-\tau)d\tau + \mathbf{v}(t),
  \label{eq:linearmimo}
\end{align}
onde $H(t)$ é uma matriz $N\times M$ que descreve a resposta dos $MN$ canais do sistema MIMO.
Os ganhos ao usar MIMO são dependentes da informação adicional dada por $\mathbf{H}$, de facto, se esta for mal condicionada, as vantagens são limitadas. 
Nas telecomunicações, procura-se explorar a diversidade espacial dada pelos $MN$ canais, o que pode ser visto como garantir redundâncias, para pelo menos um caminho esteja disponível.
Aliás, antenas recetoras distanciadas de apenas um comprimento de onda podem receber sinais completamente independentes que transportam informação redundante.

No caso de um radar MIMO coerente e um único alvo, as antenas estão tão perto que observam a mesma reflectividade, e a única diferença será um desvio de fase que está relacionado com o ângulo de incidência.

Embora existam muitas semelhanças entre radar e telecomunicações MIMO, a meta nas comunicações é maximizar SNR no recetor, enquanto que para o radar é optimizar as outras propriedades da antena, eventualmente com o custo de reduzir SNR.
A análise seguinte mostra como o processamento coerente dos sinais de um sistema de radar MIMO pode ser usado para melhorar o desempenho do sistema.

\subsection{Agregado Virtual MIMO}

Um radar MIMO que transmite formas de onda ortogonais pode ser analisado como um agregado virtual.
Trata-se de usar $M$ elementos a transmitir e $N$ elementos usados na recepção, tal que se têm $NM$ elementos virtuais.

Vamos considerar um agregado uni-dimensional, com a posição relativa de cada elemento em relação a um centro arbitrário representada por $x$. 
Adicionalmente, $x_T$ e $x_R$ denotam um elemento transmissor e recetor, respetivamente.
O agregado virtual MIMO correspondente é:
\begin{align}
  \left\{ \frac{x_{Tm}+x_{Rn}}{2}:\; m=1,\ldots,M;\,n=1,\ldots,N \right\}.
  \label{eq:virtualmimo}
\end{align}
Esta construção é feita considerando cada par pseudo-biestático\footnote{O termo pseudo-biestático significa no caso de MIMO coerente que o ângulo biestático é assumido ser pequeno, ie, os elementos transmissores e recetores estão próximos.} entre elementos transmissores e recetores, notando que cada emissor deve utilizar formas de ondas ortogonais.

Na figura \ref{fig:virt1} \emph{a)} observa-se o agregado de fase, em que as formas de onda estão perfeitamente correlacionadas, o que implica existirem $N$ centros de fase.
Na figura \ref{fig:virt1} \emph{b)}, usam-se formas de onda não correlacionadas, o que produz $NM$ centros de fase virtuais, embora não todos distintos.
Se o objetivo for obter um agregado virtual contínuo, tal é possível separando os elementos transmissores, como na figura \ref{fig:virt1} \emph{c)}. 

\begin{figure}[h]
  \centering
  \includegraphics[height=7cm]{virt1.png}
  \caption{Agregados virtuais equivalentes.}
  \label{fig:virt1}
\end{figure}

Como observado, transmitir formas de onda ortogonais origina centros de fase adicionais, o que traz duas vantagens.
Em primeiro lugar, o \textit{spatial sampling rate}\todo{Preciso de tradução disto e talvez incluir uma definição} pode ser aumentado por um fator de $M$ face ao agregado de fase, o que é útil para imagiamento com abertura sintética.\todo{Verificar esta tradução}
Em segundo, exitem benifícios ao nível da resolução angular quando se aumenta o tamanho do agregado virtual.

\todo[inline]{Subsecção sobre coarray, se me apetecer}

\section{Processamento de Sinais para radar MIMO}

\section{Exemplos de Aplicações}

\subsection{Aplicação a Sistemas de Imagens por Radar}

A captação de imagens de grandes superfícies geográficas de alta resolução é um campo de elevada relevância e dificuldade devido à escala considerada e a atenuação óptica da atmosfera para grandes distâncias e altitudes necessárias para captação eficiente das superfícies mencionadas.

Desde cedo se relevou uma aplicação útil da tecnologia radar. Colocando radares a altitudes elevadas para o propósito, os sinais emitidos são refletidos pela superfície terrestre e são captados pelo radar com propriedades diferentes dependendo da superfície atingida, permitindo ao radar obter informação sobre a composição de determinadas zonas captadas no seu alcance.

Em comparação a fotografia óptica, que para a escala considerada tem o seu alcance diminuído não só por obstáculos atmosféricos como nuvens, tempestades, ou outros fenómenos meteorológicos, como também pela composição da atmosfera em si, os sinais de rádio emitidos por radares são muito menos afetados por este tipo de interferências.

Devido a estas vantagens, a criação de imagens de grandes superfícies e a sua aplicação em mapeamento terrestre, monitorização ambiental e sistemas de reconhecimento militar é frequentemente auxiliada por sistemas de radares próprios para o efeito.

\subsubsection{Radar de Abertura Sintética}

O \textsc{Radar de Abertura Sintética}, doravante referido pela sigla inglesa SAR, é um sistema de radar aéreo ou espacial que beneficia do seu movimento para obter vários sinais diferentes ao longo da sua trajetória, juntando-os para criar um resultado final. \todo{Falta relacionar tudo com o MIMO}

\printbibliography
\listoftodos
\end{document}
