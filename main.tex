% arara: xelatex: { synctex: yes, shell: yes }
\RequirePackage[l2tabu, orthodox]{nag}
\documentclass[palatino,portuguese]{ist-report}

\usepackage{csquotes}
\usepackage{fvextra}
\usepackage{biblatex}
\addbibresource{main.bib}

\usepackage{siunitx}

\usepackage{amssymb}
\usepackage{textcomp}
\usepackage{gensymb}

\usepackage{booktabs}

\setgraphicspath{{graphics/}}

\author{Daniel de Schiffart \\ \texttt{81479} \and João Gonçalves \\ \texttt{81040}}
\title{Radar MIMO}
\subtitle{}
\course{Mestrado Integrado em Engenharia Aeroespacial}
\subject{Sistemas de Radar}

\begin{document}

\makecover

{\hypersetup{linkcolor={black}} \tableofcontents}

\begin{abstract}
	Neste trabalho vamos tratar de radares multi-estáticos MIMO, acrónimo para \textit{Multiple-Input Multiple-Output}.
\end{abstract}

\end{document}
